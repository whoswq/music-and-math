\documentclass[a4paper]{ctexart}
\usepackage[top=2.3cm,bottom=2cm,left=1.7cm,right=1.7cm]{geometry} 
\usepackage{amsmath, amssymb}
\usepackage{color}
\usepackage{listings}
\usepackage{mathrsfs} 
\usepackage{booktabs}
\usepackage{amsthm}
\usepackage{longtable} 
\usepackage{graphicx}
\usepackage{subfigure}
\usepackage{caption}
\usepackage{fontspec}
\usepackage{titlesec}
\usepackage{fancyhdr}
\usepackage{latexsym}
\usepackage{subfigure}
\usepackage{cite}
\CTEXsetup[format={\Large\bfseries}]{section}
\def\d{\mathrm{d}}
\def\e{\mathrm{e}}
\newcommand{\mb}[1]{\mathbf{#1}}
\newcommand{\mr}[1]{\mathrm{#1}}
\newcommand{\dv}[2]{\frac{\d{#1}}{\d{#2}}}
\newcommand{\pdv}[2]{\frac{\partial{#1}}{\partial{#2}}}
\def\degree{$^{\circ}$}
\title{\textbf{鼓的振动模态}}
\author{最终提交前需要修改;王崇斌\;1800011716}
\date{}
% 注意,重新定义了微分与偏微分的表达,使之更加简洁
\makeatletter %使\section中的内容左对齐
\makeatother
\begin{document}
	\pagestyle{fancy}
	\pagestyle{fancy}
    \lhead{音乐与数学期末课题}
	\chead{}
	\rhead{\today}
	\maketitle
	\thispagestyle{fancy}
	\section{鼓与振动}
	\subsection{鼓的构造}
	鼓是一种由近似圆柱形的鼓身和鼓身一端或两端蒙上拉紧的膜构成的打击乐器。
	演奏时通常是演奏者击打圆形的鼓面,鼓面振动发出声音,可以将发声过程看作
	圆形薄膜的振动过程。但是真实的鼓发声过程并不简单地对应于一个边界固定的
	圆形薄膜的自由振动(没有耗散,没有外力),比如我们忽略了鼓面的能量耗散:
	鼓面与空气的相互作用,鼓面本身的阻尼等等。想要建立准确的符合实际的模型,
	首先需要在物理上对鼓进行分类。
	\par 鼓作为振动系统可以分为3类\cite{a}:
	1.由单个薄膜与封闭的空气室构成(kettledrums),2.由单个薄膜和两侧均开放的空气室构成(tom-toms, congas),
	3.由两片薄膜与中间封闭的空气室耦合而成(bass drums, snare drums)。可以看到,
	真实的情况远比理想的薄膜振动要复杂,不仅涉及到鼓面与空气作用导致的能量耗散,甚至
	涉及到两个鼓面振动的耦合。
	\subsection{膜自由振动波动方程的建立}
	要想精确描述一个振动过程,必须找到与之对应的波动方程。但是真实的情况非常复杂,
	只能先从理想的情况开始讨论。
	我们\textbf{假设}膜上的每一点都在垂直于薄膜平面(平衡位置)的方向上做小振动,
	这样用一个标量可以完全描述每一点的运动情况,得到的波动方程是标量波动方程。
	用坐标$(x, y)$来标记平衡时每一点的位置,用$u(x, y)$来标记与平衡位置的偏离。
	\textbf{假设}薄膜没有厚度,薄膜的形变不能带来法向的张力,
	即膜的内部只有平行于薄膜的张力(这个假设与柔软的弦是一样的),
	很显然这个假设对于平板(和细杆)是不成立的,也就是说将要得到的波动方程
	并不能描述与它们的振动。
	\par 建立波动方程的思路是考虑一小块薄膜的受力情况,写出它对应的运动方程。如下图
	所示,OABC表示偏离平衡位置的薄膜,O点的坐标为$(x, y)$。假设这个薄膜的应力张量
	为$\sigma$,在最一般的情况下也是位置与时间的函数。
	\begin{figure}[htbp]
		\centering
		\includegraphics[scale=0.2]{force.png}
		\caption{受力分析}
	\end{figure} 
	可以通过应力张量表示出这个小薄膜的每个边的受力情况:
	\begin{align}
		F_{\mr{AB}, x} &= \int_{\mr{A}}^{\mr{B}} \sigma_{xy}\d x + \sigma_{xx}\d y \approx \sigma_{xx}(x+\Delta x, y)\Delta y\\
		F_{\mr{AB}, y} &= \int_{\mr{A}}^{\mr{B}} \sigma_{yy}\d x + \sigma_{yx}\d y \approx \sigma_{yx}(x+\Delta x, y)\Delta y\\
		F_{\mr{OC}, x} &= \int_{\mr{O}}^{\mr{C}} \sigma_{xy}\d x + \sigma_{xx}\d y \approx \sigma_{xx}(x, y)\Delta y\\
		F_{\mr{OC}, y} &= \int_{\mr{O}}^{\mr{C}} \sigma_{yy}\d x + \sigma_{yx}\d y \approx \sigma_{yx}(x, y)\Delta y\\
		F_{\mr{OA}, x} &= \int_{\mr{O}}^{\mr{A}} \sigma_{xy}\d x + \sigma_{xx}\d y \approx \sigma_{xy}(x, y)\Delta x\\
		F_{\mr{OA}, y} &= \int_{\mr{O}}^{\mr{A}} \sigma_{yy}\d x + \sigma_{yx}\d y \approx \sigma_{yy}(x, y)\Delta x\\
		F_{\mr{CB}, x} &= \int_{\mr{C}}^{\mr{B}} \sigma_{xy}\d x + \sigma_{xx}\d y \approx \sigma_{xy}(x, y + \Delta y)\Delta x\\
		F_{\mr{CB}, y} &= \int_{\mr{C}}^{\mr{B}} \sigma_{yy}\d x + \sigma_{yx}\d y \approx \sigma_{yy}(x, y + \Delta y)\Delta x
	\end{align}
	\par 上面的约等于在薄膜很小的情况下成立,忽略了一些在后面计算二阶导数时可以忽略的高阶小量。
	下面将计算每个边的受力在$z$方向上的分量(因为我们假设只在$z$方向上有运动),
	\textbf{假设}薄膜只做微小的振动,在这种情况下,$x$方向的力在$z$方向的贡献可以用
	薄膜沿$x$方向的切线与$x$轴夹角的正切值,也就是$\pdv{u}{x}$来衡量;同样的,$y$方向的
	力对$z$方向的力的贡献可以用$\pdv{u}{y}$来衡量。那么AB,\, CO两个边对于$z$方向的力的
	贡献为:
	\begin{align}
		F_{1} =& \,\sigma_{xx}(x+\Delta x, y)\Delta y\pdv{u}{x}(x+\Delta x, y) + \sigma_{yx}(x+\Delta x, y)\Delta y\pdv{u}{y}(x+\Delta x, y)\\
		&- \sigma_{xx}(x, y)\Delta y\pdv{u}{x}(x, y) + \sigma_{yx}(x, y)\Delta y\pdv{u}{y}(x, y)\\
		\approx& \, \left[\pdv{}{x}\left(\sigma_{xx}\pdv{u}{x}\right) + \pdv{}{x}\left(\sigma_{yx}\pdv{u}{y}\right)\right]\Delta x\Delta y
	\end{align}
	\par 同理OA,\, CB两个边对于$z$方向力的贡献为:
	\begin{align}
		F_{2} \approx  \left[\pdv{}{y}\left(\sigma_{xy}\pdv{u}{x}\right) + \pdv{}{y}\left(\sigma_{yy}\pdv{u}{y}\right)\right]\Delta x\Delta y
	\end{align}
	\par 那么,根据Newton运动定律,薄膜微元的振动方程可以写为:
	\begin{align}
		&F_{1} + F_{2} = \rho \Delta x\Delta y \pdv{^{2}u}{t^{2}}\\
		\Rightarrow& \left[\pdv{}{x}\left(\sigma_{xx}\pdv{u}{x}\right) + \pdv{}{x}\left(\sigma_{yx}\pdv{u}{y}\right)\right] + 
		\left[\pdv{}{y}\left(\sigma_{xy}\pdv{u}{x}\right) + \pdv{}{y}\left(\sigma_{yy}\pdv{u}{y}\right)\right] = \rho\pdv{^2 u}{t^{2}}
	\end{align}
	\par 这是一般的没有能量耗散的薄膜小振动时所满足的方程,其中应力张量可以是位置
	和时间的函数,在小振动假设下,一般认为其只是位置的函数。考虑到应力张量总是一
	个对称张量\cite{b},可以将上面的波动方程表示为一个更加紧凑的形式,其中使用了求和约定:
	\begin{align}
		\pdv{}{x^{i}}\left(\sigma_{ij}\pdv{u}{x^{j}}\right) = \rho\pdv{^2 u}{t^2}
	\end{align}
	
	进一步\textbf{假设}薄膜在平衡时内部的应力张量是一个常量(与位置无关,比如在各个
	方向均匀地拉伸一个各向同性方形薄膜),同时考虑到这是一个对称张量,那么可以选取
	一个特殊的直角坐标系使得每一点处的张量都具有对角的形式,方程可以化简为:
	\begin{align}
		\sigma_{xx}\pdv{^{2}u}{x^{2}} + \sigma_{yy}\pdv{^2 u}{y^2} = \rho \pdv{^{2}u}{t^{2}}
	\end{align}
	\par 如果进一步假设薄膜是各向同性的,应力张量为一个常数乘上单位张量
	那么薄膜的小振动方程在任意一个直角坐标系下都可以写成:
	\begin{align}
		\pdv{^2 u}{x^2} + \pdv{^{2}u}{y^{2}} = \frac{1}{c^{2}}\pdv{^2 u}{t^2}\quad\quad\quad c^2 = \frac{\rho}{\sigma}
	\end{align}
	\par 其中$c$称为波速。将要求解的就是这个波动方程,这是一个理想化的模型,我们做了足够多的假设。
	\section{固定边界的圆膜自由振动}
	\subsection{问题的描述}
	由于鼓面的四周是完全固定的,在考虑初始条件后可以构成一个定解问题。假设
	圆膜的半径为$a$,给定初始时刻时圆膜上每一点的位移和速度,定解问题可以写为:
	\begin{align}
		\left\{
			\begin{array}{lr}
				\displaystyle\frac{1}{c^2}\pdv{^{2}u}{t^2} = \nabla^{2}u\\
				u|_{x^2 + y^2 = a^2} = 0\\
				u|_{t=0} = f(x, y),\, \pdv{u}{t}|_{t=0} = g(x, y)
			\end{array}
		\right.
	\end{align}
	\par 由于求解区域的特殊性,考虑将方程转化到极坐标中,同时要转换方程与定解条件:
	\begin{align}
		\left\{
			\begin{array}{lr}
				\displaystyle\frac{1}{c^2}\pdv{^{2}u}{t^2} = \frac{1}{r}\pdv{}{r}\left(r\pdv{u}{r}\right) + \frac{1}{r^2}\pdv{^2 u}{\phi^{2}}\\
				u|_{\phi=0} = u|_{\phi=2\pi},\, \pdv{u}{\phi}|_{\phi=0} = \pdv{u}{\phi}|_{\phi=2\pi}\\
				u|_{r=0}\,\mr{finit},\, u|_{r=a}=0\\
				u|_{t=0} = F(r, \phi),\, \pdv{u}{t}|_{t=0} = G(r, \phi)
			\end{array}
		\right.
	\end{align}
	\par 注意到将直角坐标转换成极坐标时,人为地增添了一些边界条件,这些边界条
	件在物理上很好理解:首先在直角坐标系中原点并不是奇点,因此转换为极坐标后我们
	也得要求原点的解不能发散(不是奇点);其次这个解应该是角度的周期函数,
	因为转过$2\pi$角度后回到圆膜上同一个点,同一个点的振动是完全相同的。
	\par 也可以给出从数学角度的理解(个人理解,其中可能包含不严谨
	的语言):直角坐标与极坐标之间的变换将一个圆形和一个长方形对应了起来,
	但是这个“对应”的性质并不总是很好,首先圆心处映射并不是单射,圆心对应于
	长方形的一个边;其次边界并不对应边界,圆形的边界对应于长方形的一个边,
	圆形区域内部的一条线(规定$\phi=0$的那条线)对应于长方形的两个边。
	这个映射破坏了原本圆形的拓扑,相当于挖去了圆心后再沿圆心处将圆形剪开,
	再把这个图形“连续地”映射成了一个长方形。在计算积分时,由于“性质不好”的点
	对于积分的贡献为零,所以可以使用这个变换计算积分。但是讨论区域上的函数时,
	区域的性质会影响其上函数的性质,最直接的就是影响连续性和可微性,上面的边界条件
	相当于将定义在矩形上的函数“沿着某个边粘起来”,使之可以定义在圆形的区域上。
	\par 将要求解的是转化到极坐标后,与原来问题等价的方程。
	\subsection{没有耗散的情形}
	上面导出的方程对应于没有耗散的情形。首先假设方程具有满足$r\, \phi$方向
	边界条件(不一定满足初始条件)的分离变量形式的特解:
	\begin{align}
		u(r, \phi, t) = v(r, \phi)T(t)
	\end{align}
	\par 将所有分离变量形式的解\textbf{线性组合}使其满足\textbf{初始条件}
	就可以得到定解问题的解。将这样的解代入方程:
	\begin{align}
		\frac{1}{c^2}\frac{T''(t)}{T} = \frac{1}{rv(r,\phi)}\pdv{}{r}\left(r\pdv{v}{r}\right)
		 + \frac{1}{r^2 v(r,\phi)}\pdv{^2 V}{\phi^2}
	\end{align}
	\par 方程两边没有相同的自变量,因此只能等于同一个常数,
	令其为$-\lambda$,得到两个常微分方程:
	\begin{align}
		&T{''}(t) + \lambda c^2T(t) = 0\\
		&\frac{1}{r}\pdv{}{r}\left(r\pdv{v}{r}\right) + \frac{1}{r^2}\pdv{^2 v}{\phi^2} + \lambda v = 0
	\end{align}
	\par 其中$v$满足的$r, \phi$方向上的边界条件与$u$的相同,构成了一个偏微
	分方程本征值问题,求解出所有的本征值$\lambda$与本征函数就可以得到圆膜振动
	的驻波解,即得到了圆膜的每一个振动\textbf{频率}和对应的\textbf{振动模态},
	将这样的驻波解线性组合可以得到满足对应任意初始条件的波动方程的解,即圆膜
	的任何一种振动都可以描述为不同振动模态以相应频率振动的线性组合,因此
	只要研究清楚每一个振动模态的行为,就可以完全描述圆膜的振动。
	\par 为了求解偏微分方程的本征值问题,进一步考虑$v$也有分离变量形式
	的满足边界条件的解$v(r, \phi) = R(r)\Phi(\phi)$,带入方程得到:
	\begin{align}
		\frac{r}{R}\dv{}{r}\left(r\dv{R}{r}\right) + \lambda r^{2} = -\frac{1}{\Phi}\dv{^2 \Phi}{\phi^2} = \mu
	\end{align}
	\par 通过$v(r, \phi)$满足的边界条件得到$R(r), \Phi(\phi)$满足的边界条件(直接带入):
	\begin{align}
		&\left\{
			\begin{array}{lr}
				R(r)\Phi(0) = R(r)\Phi(2\pi)\\
				R(r)\Phi'(0) = R(r)\Phi'(2\pi)
			\end{array}
		\right.\\
		&\left\{ 
			\begin{array}{lr}
				R(0)\Phi(\phi)\quad \mr{finit}\\
				R(0)\Phi(\phi) = 0
			\end{array}
		\right.
	\end{align}
	\par 考虑到$R,\, \Phi$均不恒等于0(否则只能给出0解),
	这些边界条件与分离变量产生的两个微分方程构成了常微分方程的本征值问题:
	\begin{equation}
		\left\{ 
			\begin{array}{lr}
				\displaystyle\frac{1}{r}\dv{}{r}\left(r\dv{R}{r} \right) + \left(\lambda - \frac{\mu}{r^2}\right)R = 0\\
				R(0)\quad\mr{finit}\\
				R(a) = 0
			\end{array}
		\right.
		\label{non_reduced Bessel equation}
	\end{equation}
	\begin{equation}
		\left\{ 
			\begin{array}{lr}
				\displaystyle\dv{^2\Phi}{\phi^2} + \mu \Phi = 0\\
				\Phi(0) = \Phi(2\pi)\\
				\Phi'(0) = \Phi'(2\pi)
			\end{array}
		\right.
		\label{angle equation}
	\end{equation}
	\par 本征值问题[\ref{angle equation}]的解这里不仔细求解(容易写出方程的通解,
	仔细讨论满足边界条件的非平凡解就可以得到本征值和本征函数),直接给出结论(
	可以参考\cite{mathematicalmethod}第15章第4节):
	\begin{align}
		\left\{ 
			\begin{array}{lr}
				\mu = m^{2}\\
				\Phi_{m1} = \sin(m\phi)\\
				\Phi_{m2} = \cos(m\phi)\\
				m = 0,1,2,\cdots 
			\end{array}
		\right.
	\end{align}
	\par 这里稍微注意一下对于同一个本征值$m^2 \neq 0$对应两个简并的本征函数;当$m^2=0$时,
	仅对应一个本征函数$\Phi_{02} = 1$。
	\par 对于本征值问题[\ref{non_reduced Bessel equation}],首先将其转化为标准形式,
	令$x = \sqrt{\lambda}r,\, y = R(r)$,本征值问题转化为一个标准形式:
	\begin{align}
		\left\{ 
			\begin{array}{lr}
				\displaystyle\frac{1}{x}\dv{}{x}\left(x\dv{y}{x} \right) + \left(1 - \frac{m^2}{x^2}\right)y = 0\\
				y(0)\quad\mr{finit}\\
				y(\sqrt{\lambda}a) = 0\\
				m = 0, 1, 2, \cdots
			\end{array}
		\right.
	\end{align}
	\par 
	这是一个整数阶Bessel方程,其通解可以写为(参考\cite{mathematicalmethod}
	第17章第1节):
	\begin{align}
		y(x) = AJ_{m}(x) + BN_{m}(x)
	\end{align}
	\par 其中$J_{m}(x),\,N_{m}(x)$为m阶Bessel函数与m阶Neumann函数。由于整数阶Neumann
	函数在0处发散,所以为了满足$x=0$处函数值有界的边界条件,只能要求$B=0$。
	带入另一个边界条件可以得到:
	\begin{align}
		J_{m}(\sqrt{\lambda}a) = 0
	\end{align}
	\par 这意味着需要知道Bessel函数的零点进而得到可取的$\lambda$值(本征值),用
	$\mu_{i}^{(m)}$表示m阶Bessel函数的零点(i是对于不同零点的一个标记),$\lambda$
	满足的条件为:
	\begin{align}
		\sqrt{\lambda} a &= \mu_{i}^{(m)}\\
		\Rightarrow \lambda &= \left(\frac{\mu_{i}^{(m)}}{a}\right)^2 
	\end{align}
	可以证明Bessel函数的零点均在实轴上\cite{specialfunction},
	对于整数阶Bessel函数,由于
	\subsection{有阻尼的情形}                                                                                                                                                                                                                                                                                                                                                                                                                                                                                                                                                                                                                                                                                                                                                                                                                                                                                                                                                                                                                                                                                                                                                                                                                                                                                                                                                                                                                                                                                                                                                                                                                                                                                                                                                                                                                                                                                                                                                                                                                                                                                                                                                                                                                                                                                                                                                                                                                                                                                                                                                                                                                                                                                                                                                                                                                                                                                                                                                                                                                                                                                                                                                                                                                                                                                                                                                                                                                                                                                                                                                                  
	\section{定音鼓的发声机制}
	\section{有固定音高乐器与无固定音高乐器}
	\section{致谢}
	\bibliographystyle{unsrt}
	\bibliography{ref}
\end{document}
